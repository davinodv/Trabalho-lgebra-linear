
\documentclass[a4paper,12pt]{article}
\usepackage[utf8]{inputenc}
\usepackage[brazil]{babel}
\usepackage{amsmath}
\usepackage{graphicx}
\usepackage{geometry}
\geometry{a4paper, margin=2cm}
\usepackage{setspace}
\setstretch{1.5}

\title{Análise de Similaridade entre Textos com TF-IDF}
\author{Arthur Davino Rizzo}
\date{FATEC São Vicente - 2025}

\begin{document}
\maketitle

\section{Introdução}
Este trabalho tem como objetivo aplicar conceitos de Álgebra Linear e Ciência de Dados por meio da técnica TF-IDF (\textit{Term Frequency-Inverse Document Frequency}) para medir a similaridade entre textos. O método é amplamente utilizado em mineração de texto, recuperação de informação e processamento de linguagem natural.

\section{Descrição do Conjunto de Dados}
O conjunto de dados consiste em 20 textos curtos, sendo 10 relacionados ao tema ``Futebol'' e 10 relacionados ao tema ``Tecnologia no Esporte''. O objetivo é verificar, por meio da análise vetorial, quais textos apresentam maior semelhança em termos de conteúdo e vocabulário.

\section{Metodologia}
A abordagem segue os seguintes passos principais:
\begin{enumerate}
    \item Vetorização dos textos com TF-IDF, transformando cada documento em um vetor numérico.
    \item Cálculo da similaridade de cosseno entre os vetores, permitindo medir a proximidade semântica.
    \item Conversão da similaridade em ângulos, de forma que ângulos menores indiquem textos mais semelhantes.
\end{enumerate}

\section{Resultados}
A matriz de similaridade e a matriz de ângulos foram obtidas a partir dos 20 textos. Cada elemento da matriz representa o grau de semelhança entre dois textos.

Os resultados mostraram que os textos sobre o mesmo tema tendem a ter alta similaridade entre si. Em especial, textos que mencionam ``análise de desempenho'' e ``big data'' apresentaram forte correlação, indicando relação temática direta.

\section{Conclusão}
O experimento demonstrou que o TF-IDF é uma ferramenta eficaz para mensurar semelhanças textuais e que o conceito de ângulo entre vetores, originário da Álgebra Linear, pode ser aplicado na interpretação de dados textuais. O método permite identificar relações semânticas e padrões de conteúdo de maneira quantitativa e objetiva.

\section{Código Fonte}
O código Python completo encontra-se no arquivo \texttt{tfidf\_text\_similarity.py}, que acompanha este relatório.

\end{document}
